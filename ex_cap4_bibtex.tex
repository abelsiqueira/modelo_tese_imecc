\chapter{Título 4}
Esse é o terceiro capítulo da sua tese.

\section{Referência bibliográfica}
Para a referência bibliográfica é utilizado o BibTeX.

\subsection{BibTeX}
BibTeX é o nome de um formato de ``banco de dados'' para referências
bibliográficas desenvolvido para ser utilizado em conjunto com o LaTeX e também
do programa responsável por processar esse ``banco de dados''.

O ``banco de dados'' corresponde a um arquivo de texto com a extensão
\lstinline+.bib+. Por padrão este modelo utiliza o arquivo \lstinline+tese.bib+
que já encontra-se com algumas entradas para servirem de exemplo.

Cada referência no BibTeX segue a seguinte estrutura:
\begin{lstlisting}
@TIPO_DOCUMENTO{identificador,
campo1 = {valor do campo 1},
campo2 = {valor do campo 2},
campo3 = {valor do campo 3},
...
}
\end{lstlisting}

Para saber quais sobre os \lstinline+TIPO_DOCUMENTO+ existentes e sobre os
campos recomenda-se a documentação do BibTeX que pode ser acessada pelo comando:
\begin{lstlisting}
$ texdoc bibtex
\end{lstlisting}

Uma das grandes vantagens de se utilizar o BibTeX é que as chances de encontrar
o BibTeX de algum material na internet é extremamente alta. Tanto o Google
Scholar como o Google Books disponibilizam o BibTeX para todos os materiais
indexados em suas respectivas bases de dados.

Para que uma entrada do \lstinline+tese.bib+ seja incluído na referência
bibliográfica ele precisa ser utilizado em algum dos arquivos \lstinline+.tex+
que compõe sua dissertação/tese. Para utilizar uma referência utiliza-se
o comando \lstinline+\cite{id}+, onde
\lstinline+id+ correspode ao \lstinline+identificador+ utilizado na entrada do
BibTeX para a referência desejada.

O comando \lstinline+\cite{id}+ insere o número da referência entre colchetes,
como mostrado abaixo:
\begin{center}
  \centering
  \begin{tabular}{|l|c|}
    \hline
    Comando & Resultado \\ \hline
    \lstinline+\cite{Swa82}+ & \cite{Swa82} \\ \hline
    \lstinline+\cite{Bailey}+ & \cite{Bailey} \\ \hline
    \lstinline+\cite{Ta}+ & \cite{Ta} \\ \hline
    \lstinline+\cite{Hale}+ & \cite{Hale} \\ \hline
  \end{tabular}
\end{center}

Por último, caso deseje incluir uma referência na referência bibliográfica mas
suprimí-la ao longo do texto você deve utilizar o comando
\lstinline+\nocite{id}+.
