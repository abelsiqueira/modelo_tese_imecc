\chapter{Título 4}
Esse é o terceiro capítulo da sua tese.

\section{Referência bibliográfica}
Para a referência bibliográfica é utilizado o BibTeX e o pacote
\lstinline+biblatex+. Poderia ser utilizado apenas o BibTeX mas existem algumas
funcionalidades disponibilizadas no pacote \lstinline+biblatex+ que são muito
convenientes.

\subsection{BibTeX}
BibTeX é o nome de um formato de ``banco de dados'' para referências
bibliográficas desenvolvido para ser utilizado em conjunto com o LaTeX e também
do programa responsável por processar esse ``banco de dados''.

O ``banco de dados'' corresponde a um arquivo de texto com a extensão
\lstinline+.bib+ e cada referência no BibTeX segue a seguinte estrutura:
\begin{lstlisting}
@TIPO_DOCUMENTO{identificador,
campo1 = {valor do campo 1},
campo2 = {valor do campo 2},
campo3 = {valor do campo 3},
...
}
\end{lstlisting}

Para saber quais os \lstinline+TIPO_DOCUMENTO+ existentes e sobre os
campos recomenda-se a documentação do BibTeX que pode ser acessada pelo comando:
\begin{lstlisting}
$ texdoc bibtex
\end{lstlisting}

Uma das grandes vantagens de se utilizar o BibTeX é que as chances de encontrar
o BibTeX de algum material na internet é extremamente alta. Tanto o Google
Scholar como o Google Books disponibilizam o BibTeX para todos os materiais
indexados em suas respectivas bases de dados.

Por padrão este modelo\nocite{ModeloIMECC} utiliza o arquivo
\lstinline+tese.bib+ que inicialmente possue apenas a referência deste modelo:
\begin{lstlisting}
@Misc{ModeloIMECC,
    author = {Raniere Silva and Tiago Macedo and Junior Soares
    and others},
    title = {Modelo de Dissertacao/Tese do Instituto de
Matematica, Estatistica e Computacao Cientifica ({IMECC}) da
Universidade Estadual de Campinas ({UNICAMP})},
    year = {2013},
    url = {https://github.com/r-gaia-cs/modelo_tese_imecc},
}
\end{lstlisting}

Se desejar adicionar mais arquivos com a extensão \lstinline+.bib+, adicione no
arquivo \lstinline+configuracoes.tex+ uma linha semelhante a
\begin{lstlisting}
\addbibresource{seu_arquivo.bib}
\end{lstlisting}

Se nunca tiver utilizado o BibTeX e desejar mais exemplos de como escrever sua
referência no arquivo \lstinline+.bib+ pois não encontrou-a disponível na
internet, você pode dar uma olhada no arquivo
\lstinline+biblatex_style_samples/sample.bib+ presente neste modelo.

\subsection{\lstinline+biblatex+}
Para que uma entrada do \lstinline+tese.bib+ seja incluído na referência
bibliográfica ele precisa ser utilizado em algum dos arquivos \lstinline+.tex+
que compõe sua dissertação/tese. Para utilizar uma referência utiliza-se uma das
variantes do comando \lstinline+\cite{id}+, onde
\lstinline+id+ correspode ao \lstinline+identificador+ utilizado na entrada do
BibTeX para a referência desejada.

O comando \lstinline+\cite{id}+ insere o número da referência entre colchetes,
como mostrado abaixo:
\begin{table}[!h]
  \centering
  \begin{tabular}{lc}
    \toprule
    Comando & Resultado \\ \midrule
    \lstinline+\cite{Swa82}+ & \cite{Swa82} \\
    \lstinline+\cite{Bailey}+ & \cite{Bailey} \\
    \lstinline+\cite{Ta}+ & \cite{Ta} \\
    \lstinline+\cite{Hale}+ & \cite{Hale} \\ \bottomrule
  \end{tabular}
\end{table}

Para inserir o nome dos autores e o número da referência entre colchetes,
utiliza-se o comando \lstinline+\textcite{id}+, como mostrado abaixo:
\begin{table}[!h]
  \centering
  \begin{tabular}{ll}
    \toprule
    Comando & Resultado \\ \midrule
    \lstinline+\textcite{Swa82}+ & \textcite{Swa82} \\
    \lstinline+\textcite{Bailey}+ & \textcite{Bailey} \\
    \lstinline+\textcite{Ta}+ & \textcite{Ta} \\
    \lstinline+\textcite{Hale}+ & \textcite{Hale} \\ \bottomrule
  \end{tabular}
\end{table}

Para inserir apenas o nome dos autores utiliza-se o comando
\lstinline+\citeauthor{id}+, como mostrado abaixo:
\begin{table}[!h]
  \centering
  \begin{tabular}{ll}
    \toprule
    Comando & Resultado \\ \midrule
    \lstinline+\citeauthor{Swa82}+ & \citeauthor{Swa82} \\
    \lstinline+\citeauthor{Bailey}+ & \citeauthor{Bailey} \\
    \lstinline+\citeauthor{Ta}+ & \citeauthor{Ta} \\
    \lstinline+\citeauthor{Hale}+ & \citeauthor{Hale} \\ \bottomrule
  \end{tabular}
\end{table}

Para inserir apenas o título da referência utiliza-se o comando
\lstinline+\citetitle{id}+, como mostrado abaixo:
\begin{table}[!h]
  \centering
  \begin{tabular}{ll}
    \toprule
    Comando & Resultado \\ \midrule
    \lstinline+\citetitle{Swa82}+ & \citetitle{Swa82} \\
    \lstinline+\citetitle{Bailey}+ & \citetitle{Bailey} \\
    \lstinline+\citetitle{Ta}+ & \citetitle{Ta} \\
    \lstinline+\citetitle{Hale}+ & \citetitle{Hale} \\ \bottomrule
  \end{tabular}
\end{table}

Para inserir apenas o ano de publicação da referência utiliza-se o comando
\lstinline+\citeyear{id}+, como mostrado abaixo:
\begin{table}[!h]
  \centering
  \begin{tabular}{lc}
    \toprule
    Comando & Resultado \\ \midrule
    \lstinline+\citeyear{Swa82}+ & \citeyear{Swa82} \\
    \lstinline+\citeyear{Bailey}+ & \citeyear{Bailey} \\
    \lstinline+\citeyear{Ta}+ & \citeyear{Ta} \\
    \lstinline+\citeyear{Hale}+ & \citeyear{Hale} \\ \bottomrule
  \end{tabular}
\end{table}

Para citações múltiplas, utiliza-se os comandos \lstinline+\cites{id1,id2,id3}+
ou \lstinline+\textcites{id1,id2,id3}+, como mostrado abaixo:
\begin{table}[!h]
  \centering
  \begin{tabular}{lc}
    \toprule
    Comando & Resultado \\ \midrule
    \lstinline+\cites{Bailey,Swa82}+ & \cites{Bailey,Swa82} \\
    \lstinline+\cites{Hale,Ta}+ & \cites{Hale,Ta} \\
    \lstinline+\textcites{Bailey,Swa82}+ & \textcites{Bailey,Swa82} \\
    \lstinline+\textcites{Hale,Ta}+ & \textcites{Hale,Ta} \\ \bottomrule
  \end{tabular}
\end{table}

Por último, caso deseje incluir uma referência na referência bibliográfica mas
suprimí-la ao longo do texto você deve utilizar o comando
\lstinline+\nocite{id}+.

\section{Estilos}
O pacote \lstinline+biblatex+ oferece vários estilos por padrão, sendo alguns
deles são:
\begin{itemize}
  \item \lstinline+numeric+,
  \item \lstinline+alphabetic+,
  \item \lstinline+authoryear+,
  \item \lstinline+authortitle+,
  \item \lstinline+verbose+.
\end{itemize}

O estilo padrão deste modelo é o estilo padrão do \lstinline+biblatex+ que
corresponde ao \lstinline+numeric+. Se você desejar utilizar um outro estilo
cujo nome é \lstinline+estilo_desejado+ você precisa trocar a linha
\begin{lstlisting}
\usepackage{biblatex}
\end{lstlisting}
no arquivo \lstinline+pacotes.tex+ por
\begin{lstlisting}
\usepackage[style=estilo_desejado]{biblatex}
\end{lstlisting}

Para facilitar sua vida, no diretório \lstinline+biblatex_style_samples+ você
encontrará alguns PDFs ilustrando alguns dos estilos disponibilizados pelo
\lstinline+biblatex+. Esses arquivos PDFs encontram-se ilustrados no
Apêndice~\ref{ape:estilo_biblatex}.

Se nenhum dos estilos oferecidos lhe agradar você poderá escrever seu próprio
estilo, mas para isso terá que ler a documentação do \lstinline+biblatex+.
