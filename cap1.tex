\chapter{T\'itulo 1}
Esse \'e o primeiro cap\'itulo da sua tese.

\section{Se\c c\~ao 1.1}
Essa é uma seção da sua tese.

Um exemplo de equação na mesma linha: 
$ 1 + 1 + 1 + 1 + 1 + 1 = 6$, 
que é trivial de ser verificada.

Um exemplo de equação em destaque:
\begin{align*}
1 + 1 + 1 + 1 + 1 + 1 &= 6,
\end{align*}
não esqueça da pontuação nas equações.

Mais exemplos de equações em destaque:
\begin{align*}
1 + 1 + 1 + 1 + 1 + 1 &= 6, \\
2 + 1 + 1 + 1 + 1 &= 6, \\
3 + 1 + 1 + 1 &= 6.
\end{align*}
Novamente não esqueça da pontuação.

Para simplificações de uma expressão, você pode utilizar:
\begin{align*}
    f(x) &= 1 + 1 + 1 + 1 + 1 + 1 \\
    &= 2 + 2 + 2 \\
    &= 6.
\end{align*}
Muito cuidado com a pontuação.

\subsection{Subse\c c\~ao 1.1.1}
Essa é uma subseção da sua tese.

Para equações muito longas, você terá que dividí-la em várias linhas:
\begin{align*}
    f(x) &= 1 + 1 + 1 + 1 + 1 + 1 + 1 + 1 + 1 + 1 \\
    &\quad {}+ 1 + 1 + 1 + 1 + 1 + 1 + 1 + 1 + 1 \\
    &\quad {}+ 1 + 1 + 1 + 1 + 1 + 1 + 1 + 1.
\end{align*}
Existe outras convenções para dividir equações muito longas mas gosto da
mostrada logo acima.

Também é possível numerar equações:
\begin{align}
    f(x) &= 1.
    \label{eq:exem_unidade}
\end{align}

Além de numerar equações, e.g., \eqref{eq:exem_unidade}, também é
possível nomea-las:
\begin{align}
    g(x) &= 0.
    \tag{EIN}
    \label{eq:exem_zero}
\end{align}

Muito cuidado com o uso da referência cruzada, e.g.,
\eqref{eq:exem_unidade} e \eqref{eq:exem_zero}.

\section{Título 1.2}
Essa é outra seção da sua tese.

Vários ambientes já estão definidos como: Teorema, Conjectura, Corolário,
Definição, \ldots

\begin{thm}
Teorema, Teorema, Teorema, Teorema.
\end{thm}

\begin{con}
Conjectura, Conjecture, Conjectura, Conjectura.
\end{con}

\begin{cor}
Corolário, Corolário, Corolário, Corolário.
\end{cor}

\begin{dfn}
Definição, Definição, Definição.
\end{dfn}

Use esses ambientes de maneira sábia.
